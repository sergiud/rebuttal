\documentclass[american,version=last,fromphone,fromemail,svgnames,dvipsnames,x11names]{scrlttr2}
\usepackage[T1]{fontenc}
\usepackage[utf8]{inputenc}
\usepackage[babel]{csquotes}
\usepackage{babel}
\usepackage{graphicx}
\usepackage{microtype}
\usepackage{lipsum}
\usepackage{rebuttal}
\usepackage{varioref}
\usepackage{hyperref}
\usepackage{cleveref}

\hypersetup{colorlinks=true}

\setkomavar{fromname}{Jane Smith}
\setkomavar{fromaddress}{%
  Pattern Recognition Lab\\
  University of Erlangen-Nürnberg
}
\setkomavar{backaddress}{}
\setkomavar{fromphone}{$+49$\,(0)\,12345678}
\setkomavar{fromemail}{firstname.lastname@fau.de}
\setkomavar{place}{Erlangen}

\begin{document}

\setkomavar{subject}{Manuscript: Extended Field of View in C-Arm Computed
  Tomography for Weight-Bearing Imaging}

\begin{letter}{Medical Physics}

\opening{Dear Prof. Jones, Dear Prof. Taylor,}

We thank you for considering our submission as a research article for Medical
Physics. We also thank the reviewers for the thoughtful comments. We have
carefully revised the document and prepared a point-by-point response to the
reviewer’s comments.

The major changes to the paper are:
\begin{itemize}
  \item Clarification of the medical impact
  \item An extension of the evaluation and discussion to 3D Cone Beam CT and
    off-center positioning of the object
  \item Inclusion of more substantive figure captions
\end{itemize}

Below, you will find the reviewers’ comments in blue, our response in normal
face and additions to the manuscript in quotes and \enquote{italic face.} Note
that we omitted footnotes and references in the quotations.

We are looking forward to hearing from you.

\closing{With kind regards,}

\end{letter}

\clearpage\noindent The following pages contain a list of editor's and
reviewers' comments followed by our replies. The comments are sequentially
numbered and associated either with the editor or the corresponding reviewer.
The replies may contain references to changes in the original manuscript which
are identified by a label consisting of a running number and followed by the
label of the original comment in parentheses. The label back references the
original reviewer comment within the manuscript. For instance, the reference
\textbf{C2~(1.3)}, which is typeset in the manuscript margin, refers to the
second change stemming from the third comment of the first reviewer.

\begin{rebuttal}[Editor's Comments]
  \begin{comment}
  The topic of this manuscript is of interest. We concur with the summary review
  of the associate editor and the reviews of the referees. The authors should
  thoroughly address all of the comments, below, in a revised manuscript.
  \end{comment}

\end{rebuttal}

\begin{rebuttal}[Specific Comments]
  \begin{comment}
  \label{c:specific}
  You need to include figure captions that make your figures intelligible.
  See~\cref{a:foo} and \cref{c:foo2}.
  \end{comment}

  \begin{answer}
    The reviewer is right. We have reworked the figure captions to make the
    figures more independent of the text. Furthermore, check if lists are
    typeset correctly:
    \begin{itemize}
      \item One
      \item Two
        \begin{enumerate}
          \item Three
          \item Four
        \end{enumerate}
    \end{itemize}
  \end{answer}

  \begin{comment}
    \label{c:c2}
    Are you going to pay for color in the print article? If not, the lines are
    not intelligible in Figures 2, 5, 11. In addition to color coding for the
    digital version, you could make lines with distinct patterns (dots, dashes,
    etc.) so they are distinct in monochrome presentations.
  \end{comment}

  \begin{answer}
    We agree with this point and we are sorry that we missed this in the first
    version of the article. The article will appear as an online only version
    and is part of the special issue associated with this year’s CT Meeting
    conference. Thus there will be no additional cost associated with color
    figures. Nonetheless, we agree that all figures need to be correct in
    grey-scale as some readers might print the article. According to the
    reviewer’s comment, we have adjusted the figures by using different patterns
    and gray levels.

    (Refer also to \cref{c:specific} for more information.)
  \end{answer}
\end{rebuttal}

\lipsum[1] \addition[label=a:foo,ref=c:specific]{\lipsum[2]} \lipsum[3]

\lipsum[4] \deletion[label=a:foo1,ref=c:c2]{An unintended addition.}
and a \change[label=c:foo2,ref=c:c2]{from}{to} change. And another addition
\addition[label=a:foo3,ref=c:c2]{here}.
Also \addition[label=a:foo4,ref=c:c2]{here}.
Maybe \addition[label=a:foo5,ref=c:c2]{here}.
Additionally \addition[label=a:foo6,ref=c:c2]{here}.
Additionally \addition[label=a:foo7,ref=c:c2]{here}.

\begin{additionenv}[label=d:par,ref=c:c2]
  \begin{itemize}
    \item \lipsum[1]
  \end{itemize}
  \lipsum[2]
\end{additionenv}

Additional notes:
\begin{itemize}
  \item Multiple source references can be
    \addition[label=a:foo8,ref={c:specific,c:c2}]{specified}.
  \item If the references do not fit into a single line in the margin, you may
    force a \addition[label=a:break,ref={c:specific,c:c2},break]{break}.
\end{itemize}

\end{document}

% vim: ft=tex spell spelllang=en_us
