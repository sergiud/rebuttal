% \iffalse meta-comment
%
% Copyright 2025 Sergiu Deitsch
%
% This work may be distributed and/or modified under the
% conditions of the LaTeX Project Public License, either version 1.3
% of this license or (at your option) any later version.
% The latest version of this license is in
%   http://www.latex-project.org/lppl.txt
% and version 1.3 or later is part of all distributions of LaTeX
% version 2005/12/01 or later.
%
% \fi
%
% \iffalse
%<package>\NeedsTeXFormat{LaTeX2e}
%<package>\ProvidesPackage{rebuttal}
%<package>   [2025-12-16 v0.2.0 Markup for rebuttal letters]
%
%<*driver>
\ProvidesFile{rebuttal.dtx}

\documentclass[american,reportchangedates,svgnames]{ltxdoc}
\usepackage[abbreviations]{foreign}
\usepackage[utf8]{inputenc}
\usepackage[T1]{fontenc}
\usepackage{babel}
\usepackage{microtype}
\usepackage[all]{nowidow}
\usepackage{rebuttal}
\usepackage{hyperref}

\EnableCrossrefs
\CodelineIndex
\RecordChanges

\hypersetup{%
  pdfauthor = {Sergiu Deitsch},
  pdftitle  = {The rebuttal package},
  urlcolor  = NavyBlue,
  linkcolor = Firebrick3,
}

\begin{document}
  \DocInput{rebuttal.dtx}
\end{document}
%</driver>
% \fi
%
% \CheckSum{0}
%
% \CharacterTable
%  {Upper-case    \A\B\C\D\E\F\G\H\I\J\K\L\M\N\O\P\Q\R\S\T\U\V\W\X\Y\Z
%   Lower-case    \a\b\c\d\e\f\g\h\i\j\k\l\m\n\o\p\q\r\s\t\u\v\w\x\y\z
%   Digits        \0\1\2\3\4\5\6\7\8\9
%   Exclamation   \!     Double quote  \"     Hash (number) \#
%   Dollar        \$     Percent       \%     Ampersand     \&
%   Acute accent  \'     Left paren    \(     Right paren   \)
%   Asterisk      \*     Plus          \+     Comma         \,
%   Minus         \-     Point         \.     Solidus       \/
%   Colon         \:     Semicolon     \;     Less than     \<
%   Equals        \=     Greater than  \>     Question mark \?
%   Commercial at \@     Left bracket  \[     Backslash     \\
%   Right bracket \]     Circumflex    \^     Underscore    \_
%   Grave accent  \`     Left brace    \{     Vertical bar  \|
%   Right brace   \}     Tilde         \~}
%
%
% \changes{v0.1.0}{2024-11-01}{Initial release}
% \changes{v0.1.1}{2024-11-14}{Improved documentation}
% \changes{v0.2.0}{2025-12-16}{Improved documentation}
%
% \GetFileInfo{rebuttal.sty}
%
% \DoNotIndex{\newcommand,\newenvironment,\def,\the,\let,\if,\fi,\else,\\}
% \DoNotIndex{\newif,\ifx,\ifcase,\or}
% \DoNotIndex{\begingroup,\endgroup}
%
% \title{The \href{https://github.com/sergiud/rebuttal}{\textsf{rebuttal}} package%
%   \thanks{This document corresponds to \textsf{rebuttal}~\fileversion, dated \filedate.}}
% \author{Sergiu % Deitsch\\%
%   \href{mailto:sergiu.deitsch@gmail.com}{\texttt{sergiu.deitsch@gmail.com}}%
% }
% \date{December 16, 2025}
%
% \maketitle
%
% \begin{abstract}
% The rebuttal \LaTeX\ package provides means for writing structured journal and
% conference paper rebuttals.
% \end{abstract}
%
% \section{Introduction}
%
% A well-structured rebuttal typically consists of the following parts:
%
% \begin{enumerate}
%   \item a master list of referee comments and author's replies, and
%   \item clearly highlighted changes to the manuscript that stem from
%   reviewers' comments.
% \end{enumerate}
%
% The \textsf{rebuttal} provides markup to support authors in producing the
% above content in a consistent manner.
%
% \section{Usage}
%
% The following sections provide an overview of the package's functionality for
% structuring rebuttals and annotating manuscripts.
%
% \subsection{Structuring the Rebuttal}
%
% \DescribeEnv{rebuttal}
% \DescribeEnv{comment}
% \DescribeEnv{answer}
% The package defines the |rebuttal| environment that can contain several blocks
% that refer to editor's or specific reviewers' comments and your replies to
% referees' comments. Specifically, the environment accepts an optional
% \oarg{title} and is expected to contain a |comment| and an |answer|
% environment. The general layout looks as follows:
%
% \begin{verbatim}
%   \begin{rebuttal}[Editor's Comments]
%     \begin{comment}
%       % Reviewer's comment
%     \end{comment}
%     \begin{answer}
%       % The reply
%     \end{answer}
%   \end{rebuttal}
% \end{verbatim}
%
% \subsection{Annotating Changes to the Manuscript}
%
% \DescribeMacro{\addition}
% \DescribeMacro{\deletion}
% \DescribeMacro{\change}
% Within the manuscript, three main commands can be used to denote additions,
% deletions, or changes. The corresponding commands are
% |\addition|\oarg{options}\marg{text}, |\deletion|\oarg{options}\marg{text},
% and |\change|\oarg{options}\marg{old text}\marg{new text}. While |\addition|
% and |\deletion| require a single argument. |\change| expects two arguments,
% where the first one denotes the changed text, and the second one the new text.
%
% All three commands require specifying their labels for referencing the
% modifications using the \meta{label} option. Additionally, the \meta{ref}
% option back references the original reviewer comment and may specify multiple
% labels, \eg
%
% \begin{verbatim}
%   \addition[label=a:new,ref={c:c1,c:c2}]{new text}.
% \end{verbatim}
%
% \subsection{Annotating Multiple Paragraphs}
%
% \DescribeEnv{additionenv}
% \DescribeEnv{changeenv}
% \DescribeEnv{deletionenv}
% Additionally to the provided markup commands, the package also defines
% equivalent environments for annotating multiple paragraphs:
%
% \begin{verbatim}
%   \begin{additionenv}[label=a:par,ref=c:missing-motivations]
%     \section{New Experiment}
%     % new text
%   \end{additionenv}
%
%   \begin{changeenv}[label=ch:par,ref=c:missing-motivations]{old text}
%     \section{Improved Experiment}
%     % new text
%   \end{changeenv}
%
%   \begin{deletionenv}[label=d:par,ref=c:missing-motivations]
%     \section{Useless Discussion}
%     % old text
%   \end{deletionenv}
% \end{verbatim}
%
% \subsection{Disabling Annotations}
%
% \DescribeMacro{\rebuttalset}
% Once a manuscript has been accepted for publication, rebuttal annotations are
% no longer necessary and only the accepted changes should be typeset.
% \textsf{rebuttal} allows to disable annotations using the |\rebuttalset|
% command by disabling the \meta{marked} option, \ie,
% \begin{verbatim}
%   \rebuttalset{marked=false}
% \end{verbatim}
% which will disable the annotiations in the current group.
%
% \section{Known Limitations}
%
% Rebuttal markup cannot be used within floating environments such as |figure|
% and |table|, and the |\caption| command.
%
% \MaybeStop{\PrintChanges\PrintIndex}
%
% \section{Implementation}
%
% \subsection{Package}
%
%    \begin{macrocode}
%<*package>
%    \end{macrocode}
% Package dependencies.
%    \begin{macrocode}
\RequirePackage{calc}
\RequirePackage{environ}
\RequirePackage{etoolbox}
\RequirePackage{marginnote}
\RequirePackage{pgfkeys}
\RequirePackage{soul}
\RequirePackage{tikz}
\RequirePackage{todonotes}
\RequirePackage{ulem}
\RequirePackage{xcolor}
\RequirePackage{xstring}
%    \end{macrocode}

% \changes{v0.2.0}{2025-01-21}{Do not require loading \textsf{xcolor} with
% specific options}
%    \begin{macrocode}
\providecolor{DodgerBlue3}{rgb}{.094,.455,.804}
\providecolor{Firebrick3}{rgb}{.804,.15,.15}
\providecolor{RedViolet}{RGB}{157,0,97}

\AtEndPreamble{
  \RequirePackage{pdfcomment}
}

\ifx\counterwithin\@undefined
\RequirePackage{chngcntr}
\fi
%    \end{macrocode}
% \begin{macro}{\rebuttal@triangle}
% The following command is used to generate a triangle in the left margin.
%    \begin{macrocode}
\newcommand\rebuttal@triangle{%
  \begingroup
  \ifx\tikzexternaldisable\@undefined\else
    \tikzexternaldisable
  \fi
  \begin{tikzpicture}[x=.5em,y=.5em]
    \fill[/rebuttal/answer/label] (0,0) -- (0,1) -- (.75,.5) -- cycle;
  \end{tikzpicture}%
  \endgroup
}
%    \end{macrocode}
% \end{macro}
% \begin{macro}{\rebuttal@decbookmarklevel}
% Unless \textsf{hyperref} was not loaded, the command decrements the current
% bookmark level.
%    \begin{macrocode}
\newcommand\rebuttal@decbookmarklevel{%
  \ifx\Hy@currentbookmarklevel\@undefined\else
    \@tempcnta\Hy@currentbookmarklevel
    \advance\@tempcnta by -1 %
    \xdef\Hy@currentbookmarklevel{\the\@tempcnta}%
  \fi
}
%    \end{macrocode}
% \end{macro}

% \begin{macro}{\rebuttal@incbookmarklevel}
% Unless \textsf{hyperref} was not loaded, the command incements the current
% bookmark level.
%    \begin{macrocode}
\newcommand\rebuttal@incbookmarklevel{%
  \ifx\Hy@currentbookmarklevel\@undefined\else
    \@tempcnta\Hy@currentbookmarklevel
    \advance\@tempcnta by 1 %
    \xdef\Hy@currentbookmarklevel{\the\@tempcnta}%
  \fi
}
%    \end{macrocode}
% \end{macro}

%    \begin{macrocode}
\newif\ifrebuttal@marked
\newif\ifrebuttal@nohyperlinks
\newif\ifrebuttal@refbreak
\newif\ifrebuttal@twocolumn

\pgfkeys{
  /rebuttal/.is family,
  /rebuttal/execute on begin/.initial=,
  /rebuttal/execute on end/.initial=,
  /rebuttal/no hyperlinks/.is if=rebuttal@nohyperlinks,
  /rebuttal/no hyperlinks=false,
  /rebuttal/marked/.is if=rebuttal@marked,
  /rebuttal/marked=true,
  /rebuttal/title/font/.store in=\rebuttal@titlefont,
  /rebuttal/title/font=\normalfont\bfseries,
  /rebuttal/title/format/.code=#1:,
  /rebuttal/change/from command/.store in=\rebuttal@changefrom,
  /rebuttal/change/from command=\sout,
  /rebuttal/change/from color/.code={\colorlet{rebuttal@change@from}{#1}},
  /rebuttal/change/from color=DodgerBlue3!50,
  /rebuttal/deletion/command/.store in=\rebuttal@delete,
  /rebuttal/deletion/command=\xout,
  /rebuttal/comment/font/.store in=\rebuttal@commentfont,
  /rebuttal/comment/font=\normalfont\itshape,
  /rebuttal/comment/label/font/.store in=\rebuttal@commentlabelfont,
  /rebuttal/comment/label/font=\normalfont\rmfamily\bfseries,
  /rebuttal/answer/font/.store in=\rebuttal@answerfont,
  /rebuttal/answer/font=\normalfont,
  /rebuttal/answer/label/.store in=\rebuttal@answerlabel,
  /rebuttal/answer/label={\rebuttal@triangle},
  /rebuttal/answer/label/font/.store in=\rebuttal@answerlabelfont,
  /rebuttal/answer/label/font=\normalfont,
  /rebuttal/answer/label/.style={black!65},
  /rebuttal/reference/font/.store in=\rebuttal@referencefont,
  /rebuttal/reference/font={%
    \tiny\bfseries%
  },
  % TODO does not work within \todo
  %/rebuttal/reference/.code=,
}

\let\rebuttalanswerlabel\rebuttal@answerlabel

\newrobustcmd\rebuttal@reference[1]{%
  \begingroup
  \rebuttal@referencefont
  \begin{tabular}{@{}c@{}}
    \begingroup
      \color{rebuttal@#1}%
      \MakeUppercase{\rebuttal@id}%
      \csname therebuttal#1\endcsname
    \endgroup
    \ifrebuttal@refbreak
      \\%
    \else
      \enspace
    \fi
    \normalfont
    % Reset color (TODO Maybe allow to modify the color)
    \color{black}%
    (\pgfkeys{/rebuttal/#1/ref/all})%
  \end{tabular}%
  \endgroup
}

\newcommand\rebuttal@define@keys[2]{
  \pgfkeys{
    /rebuttal/#1/label/.initial=,
    /rebuttal/#1/label/.value required,
    /rebuttal/#1/break/.is if=rebuttal@refbreak,
    /rebuttal/#1/ref/.initial=,
    /rebuttal/#1/ref/.value required,
    /rebuttal/#1/ref/create/.code={
      \ifrebuttal@nohyperlinks
        \ref*{##1}%
      \else
        \ref{##1}%
      \fi
    },
    /rebuttal/#1/ref/all/.code={%
      \pgfkeysgetvalue{/rebuttal/#1/ref}\rebuttal@currefs
      \edef\rebuttal@tmp{\rebuttal@currefs}%
      \foreach
      [
        %expand list,
        count=\ri from 0,
      ]
      \r in \rebuttal@tmp {%
        \ifnum\ri > 0\relax
          ,\,%
        \fi
        \pgfkeys{/rebuttal/#1/ref/create=\r}%
      }%
    },
    /rebuttal/#1/color/.code={\colorlet{rebuttal@#1}{##1}},
    /rebuttal/#1/color=#2,
    /rebuttal/#1/font/.value required,
    /rebuttal/#1/font/.code={
      \expandafter\def\csname rebuttal@#1@font\endcsname{##1}%
    },
    /rebuttal/#1/font=\normalfont\mdseries,
  }
}

\rebuttal@define@keys{addition}{RedViolet}
\rebuttal@define@keys{deletion}{Firebrick3}
\rebuttal@define@keys{change}{DodgerBlue3}

\newlength\rebuttal@textwidth
\newlength\rebuttal@hskip

\newcommand\rebuttal@command[3]{%
  \newcounter{rebuttal#1}%
  \AtEndPreamble{%
    \@ifpackageloaded{cleveref}{%
      \crefname{rebuttal#1}{#1}{#1s}%
    }{}%
  }%
  % Marked version with full output
  \expandafter\newcommand\csname #1\endcsname[#3][]{%
    \ifrebuttal@marked
      \stepcounter{rebuttalchanges}%
      \refstepcounter{rebuttal#1}%
      \begingroup
        \tikzset{notestyleraw/.append style={
          inner sep=\z@,
          rounded corners=\z@,
        }}%
        \pgfqkeys{/rebuttal/#1}{##1}%
        \label{\pgfkeysvalueof{/rebuttal/#1/label}}%
        \colorlet{Changes@Color}{rebuttal@#1}%
        \begingroup
%    \end{macrocode}
% Main content.
%    \begin{macrocode}
          #2%
        \endgroup
        \StrLeft{#1}{1}[\rebuttal@id]%
        \ifx\tikzexternaldisable\@undefined\else
          \tikzexternaldisable
        \fi
%    \end{macrocode}
% Shrink to text.
%    \begin{macrocode}
        \settowidth\rebuttal@textwidth{\rebuttal@reference{#1}}%
        \tikzset{notestyleraw/.append style={text width=\rebuttal@textwidth}}%
%    \end{macrocode}
% Place labels in left margin at the same distance as on the right.
%    \begin{macrocode}
        \setlength\rebuttal@hskip{\marginparwidth}%
        \addtolength\rebuttal@hskip{-\rebuttal@textwidth}%
        \let\old@xmpar\@xmpar
%    \end{macrocode}
% Hook into \LaTeX\ kernel command since |\if@firstcolumn| is not unreliable
% because margin notes are floats.
%    \begin{macrocode}
        \renewcommand\@xmpar[2][]{%
          \old@xmpar[\hspace*{\rebuttal@hskip}####1]{####2}%
        }%
        \todo
        [
          backgroundcolor=none,
          bordercolor=none,
          linecolor=rebuttal@#1,
          %size=\tiny,
        ]{%
          \rebuttal@reference{#1}%
          %\pgfkeys{/rebuttal/reference=#1}
        }%
      \endgroup
    \else
      #2%
    \fi
  }%
}
%    \end{macrocode}

% \begin{environment}{additionenv}
% Provide an equivalent environment for the |\addition| command.
%    \begin{macrocode}
\NewEnviron{additionenv}[1][]{%
  \addition[#1]{\BODY}%
}%
%    \end{macrocode}
% \end{environment}

% \begin{environment}{changeenv}
% Provide an equivalent environment for the |\change| command.
%    \begin{macrocode}
\NewEnviron{changeenv}[2][]{%
  \change[#1]{#2}{\BODY}%
}%
%    \end{macrocode}
% \end{environment}

% \begin{environment}{deletionenv}
% Provide an equivalent environment for the |\deletion| command.
%    \begin{macrocode}
\NewEnviron{deletionenv}[1][]{%
  \deletion[#1]{\BODY}%
}%
%    \end{macrocode}
% \end{environment}

%    \begin{macrocode}
\newcommand\rebuttal@added[1]{%
  \ifrebuttal@marked
    \rebuttal@addition@font%
    \textcolor{rebuttal@addition}{#1}%
  \else
    #1%
  \fi
}

\newcommand\rebuttal@deleted[1]{%
  \ifrebuttal@marked
    \rebuttal@deletion@font%
    \textcolor{rebuttal@deletion}{\rebuttal@delete{#1}}%
  \fi
}

\newcommand\rebuttal@replaced[2]{%
  \ifrebuttal@marked
    \rebuttal@change@font%
    \textcolor{rebuttal@change@from}{\rebuttal@changefrom{#1}}%
    \allowbreak
    \textcolor{rebuttal@change}{#2}%
  \else
    #2%
  \fi
}

\rebuttal@command{addition}{\rebuttal@added{#2}}{2}
\rebuttal@command{deletion}{\rebuttal@deleted{#2}}{2}
\rebuttal@command{change}{\rebuttal@replaced{#2}{#3}}{3}

\newcounter{rebuttalcomments}
\newcounter{rebuttalchanges}

\newcounter{rebuttal}
\newcounter{rebuttalcomment}
\newcounter{rebuttalanswer}

\renewcommand\therebuttalcomment{%
  \arabic{rebuttal}.\arabic{rebuttalcomment}%
}

\let\theHrebuttalcomment\therebuttalcomment

\renewcommand\therebuttalanswer{%
  \therebuttalcomment%
}

\counterwithin*{rebuttalcomment}{rebuttal}%
\counterwithin*{rebuttalanswer}{rebuttalcomment}%

\newdimen\rebuttal@labelsep
%    \end{macrocode}

% \begin{macro}{\rebuttal@startlinkguard}
% Disable hyperlinks if |NoHyper| environment is available.
%    \begin{macrocode}
\newcommand\rebuttal@startlinkguard{%
  \ifx\NoHyper\@undefined\else
    \ifrebuttal@nohyperlinks
      \begin{NoHyper}%
    \fi
  \fi
}
%    \end{macrocode}
% \end{macro}

% \begin{macro}{\rebuttal@startlinkguard}
% End of disable hyperlinks if |NoHyper| environment is available.
%    \begin{macrocode}
\newcommand\rebuttal@endlinkguard{
  \ifx\endNoHyper\@undefined\else
    \ifrebuttal@nohyperlinks
      \end{NoHyper}%
    \fi
  \fi
}
%    \end{macrocode}
% \end{macro}

% TODO Comment labels tooltips
% \begin{environment}{rebuttal}
% The main environment that defines the remaining ones.
% \changes{v0.1.1}{2024-11-14}{Fixed incorrect bookmark structure}
%    \begin{macrocode}
\newenvironment{rebuttal}[1][]{%
  \rebuttal@startlinkguard
%    \end{macrocode}
% Execute user code.
%    \begin{macrocode}
  \pgfkeysvalueof{/rebuttal/execute on begin}%
  \begingroup
  \refstepcounter{rebuttal}%
  \if@twocolumn
    \onecolumn
%    \end{macrocode}
% Remember the layout.
%    \begin{macrocode}
    \rebuttal@twocolumntrue
  \else
    \rebuttal@twocolumnfalse
  \fi
  \begingroup
  \par
  \addvspace{\bigskipamount}%
  \ifstrempty{#1}{}{%
%    \end{macrocode}
% In case the current bookmark level is 0, we leave it as is. Otherwise, we
% increment the level to anchor the following bookmarks below the current
% bookmark. This ensures that the bookmark structure is correct even if the
% document either does not use or provides structuring commands (\eg,
% missing |\section| in a \textsf{scrlttr2} document).
%    \begin{macrocode}
    \ifx\Hy@currentbookmarklevel\@undefined\else
      \ifnum\Hy@currentbookmarklevel>0
        \@tempcnta\Hy@currentbookmarklevel
        \advance\@tempcnta by 1 %
        \xdef\Hy@currentbookmarklevel{\the\@tempcnta}%
      \fi
    \fi
    \ifx\currentpdfbookmark\@undefined\else
      \currentpdfbookmark{#1}{rebuttalcomment.\therebuttalcomment}%
    \fi
    \noindent\bgroup\rebuttal@titlefont\strut
    \pgfkeys{/rebuttal/title/format=#1}\egroup%
  }%
  \par%
  \@afterheading
  \vspace*{\medskipamount}%
  \begin{trivlist}%
  \let\rebuttal@oldmakelabel\makelabel%
  \renewcommand\makelabel[1]{\llap{\rebuttal@commentlabelfont ##1}}%
  \labelwidth\z@
  %\setlength{\parsep}{\smallskipamount}%
  %\setlength{\labelsep}{1pt}%
  %\setlength{\topsep}{\z@}%
  %\setlength{\partopsep}{\bigskipamount}
  \let\comment\@undefined
  \let\endcomment\@undefined
  \newenvironment{comment}{%
    \rebuttal@incbookmarklevel
    \apptocmd\@item{%
      \ifx\currentpdfbookmark\@undefined\else
        \currentpdfbookmark{Comment~\therebuttalcomment}{rebuttalcomment.\therebuttalcomment}%
      \fi
    }{%
      \PackageInfo{rebuttal}{The \string\@item\space command was patched to
        provide comment bookmarks.}%
    }{%
      \PackageWarning{rebuttal}{Failed to patch the \string\@item\space command.
        Bookmarks may not be displayed correctly.}%
    }%
%    \end{macrocode}
% Do not allow to place answers within a comment by undefining the answer
% environment
%    \begin{macrocode}
    \let\answer\@undefined
    \let\endanswer\@undefined
    \stepcounter{rebuttalcomments}%
    \refstepcounter{rebuttalcomment}
    %\settowidth{\labelwidth}{\rebuttal@commentlabelfont #1}
    %\addtolength{\labelwidth}{1em}
    \item
    [%
      \therebuttalcomment%
    ]%
    \begingroup
    \rebuttal@commentfont
    \let\makelabel\rebuttal@oldmakelabel
  }{%
    \endgroup
    \rebuttal@decbookmarklevel
  }
  \newenvironment{answer}{%
    \par\medskip
    \@afterheading
    \rebuttal@incbookmarklevel
    \begin{trivlist}
      \ifx\belowpdfbookmark\@undefined\else
        \belowpdfbookmark{Anwser to~\therebuttalcomment}{rebuttalanswer.\therebuttalanswer}%
      \fi
      \normalfont\rebuttal@answerfont%
      \rebuttal@labelsep\labelsep%
      \newcommand\rebuttal@makelabel[1]{\rebuttal@answerlabelfont ####1\enspace}
      \renewcommand\makelabel[1]{\rebuttal@makelabel{####1}}%
      \labelsep\z@
      \labelwidth\z@
      \let\makelabel\empty
      %\settowidth\labelwidth{\rebuttal@makelabel{\rebuttal@answerlabel}}%
      \item
        [%
          \ifx\pdftooltip\@undefined
            \llap{\rebuttal@answerlabel\enspace}%
          \else
            \llap{\pdftooltip{\rebuttal@answerlabel}{Reply}\enspace}%
          \fi
        ]%
        \begingroup
        \labelsep\rebuttal@labelsep
        \refstepcounter{rebuttalanswer}%
  }{%
    \endgroup
    \end{trivlist}%
    \rebuttal@decbookmarklevel
  }
}{%
  \end{trivlist}%
  \rebuttal@decbookmarklevel
  \endgroup
  \par\addvspace{\bigskipamount}%
  \ifrebuttal@twocolumn
%    \end{macrocode}
% Restore the layout.
%    \begin{macrocode}
    \twocolumn
  \fi
  \endgroup
%    \end{macrocode}
% Execute user code.
%    \begin{macrocode}
  \pgfkeysvalueof{/rebuttal/execute on end}%
  \rebuttal@endlinkguard
}
%    \end{macrocode}
% \end{environment}

% Support \textsf{cleveref}.
%    \begin{macrocode}
\AtEndPreamble{%
  \@ifpackageloaded{cleveref}{%
    \crefname{rebuttalcomment}{comment}{comments}
    \crefname{rebuttalanswer}{answer}{answers}
    \Crefname{rebuttalcomment}{Comment}{Comments}
    \Crefname{rebuttalanswer}{Answer}{Answers}
  }{}
}

\newcommand\rebuttaltotalcomments{%
  \textbf{??}%
  \@latex@warning{Reference \string\rebuttaltotalcomments\space on page \thepage\space undefined}%
}

\newcommand\rebuttaltotalchanges{%
  \textbf{??}%
  \@latex@warning{Reference \string\rebuttaltotalchanges\space on page \thepage\space undefined}%
}

\AtEndDocument{%
  \immediate\write\@auxout{%
    \gdef\string\rebuttaltotalcomments{\arabic{rebuttalcomments}}%
    \gdef\string\rebuttaltotalchanges{\arabic{rebuttalchanges}}%
  }
}
%    \end{macrocode}
% \begin{macro}{\rebuttalset}\marg{options}
% Allows to update package options.
%    \begin{macrocode}
\newcommand\rebuttalset[1]{%
  \pgfqkeys{/rebuttal}{#1}%
}
%</package>
%    \end{macrocode}
% \end{macro}
%
% \subsection{Example}
%
%    \begin{macrocode}
%<*example1>
\documentclass[american,version=last,fromphone,fromemail]{scrlttr2}
%    \end{macrocode}
% Loading \textsf{xcolor} explicitly is not necessary. This line is solely used
% to ensure that the \textsf{rebuttal} package does not rely on specific
% \textsf{xcolor} options.
%    \begin{macrocode}
\usepackage[]{xcolor}
\usepackage[T1]{fontenc}
\usepackage[utf8]{inputenc}
\usepackage[babel]{csquotes}
\usepackage{babel}
\usepackage{graphicx}
\usepackage{microtype}
\usepackage{lipsum}
\usepackage{rebuttal}
\usepackage{varioref}
\usepackage{hyperref}
\usepackage{cleveref}

\hypersetup{colorlinks=true}

\setkomavar{fromname}{Jane Smith}
\setkomavar{fromaddress}{%
  Pattern Recognition Lab\\
  University of Erlangen-Nürnberg
}
\setkomavar{backaddress}{}
\setkomavar{fromphone}{$+49$\,(0)\,12345678}
\setkomavar{fromemail}{firstname.lastname@fau.de}
\setkomavar{place}{Erlangen}

\begin{document}

\setkomavar{subject}{Manuscript: Extended Field of View in C-Arm Computed
  Tomography for Weight-Bearing Imaging}

\begin{letter}{Medical Physics}

\opening{Dear Prof. Jones, Dear Prof. Taylor,}

We thank you for considering our submission as a research article for Medical
Physics. We also thank the reviewers for the thoughtful comments. We have
carefully revised the document and prepared a point-by-point response to the
reviewer’s comments.

The major changes to the paper are:
\begin{itemize}
  \item Clarification of the medical impact
  \item An extension of the evaluation and discussion to 3D Cone Beam CT and
    off-center positioning of the object
  \item Inclusion of more substantive figure captions
\end{itemize}

Below, you will find the reviewers’ comments in blue, our response in normal
face and additions to the manuscript in quotes and \enquote{italic face.} Note
that we omitted footnotes and references in the quotations.

We are looking forward to hearing from you.

\closing{With kind regards,}

\end{letter}
%%
%% Copyright (C) 2024 by Sergiu Deitsch
%%
%% This work may be distributed and/or modified under the
%% conditions of the LaTeX Project Public License, either version 1.3
%% of this license or (at your option) any later version.
%% The latest version of this license is in
%%   http://www.latex-project.org/lppl.txt
%% and version 1.3 or later is part of all distributions of LaTeX
%% version 2005/12/01 or later.
%%

\input docstrip.tex

\keepsilent
\askforoverwritefalse

\usedir{tex/latex/rebuttal}

\preamble

Copyright (C) 2018-2021, 2024 Sergiu Deitsch

This work may be distributed and/or modified under the
conditions of the LaTeX Project Public License, either version 1.3
of this license or (at your option) any later version.
The latest version of this license is in
  http://www.latex-project.org/lppl.txt
and version 1.3 or later is part of all distributions of LaTeX
version 2005/12/01 or later.

This work has the LPPL maintenance status `maintained'.

The Current Maintainer of this work is Sergiu Deitsch.

This work consists of the files rebuttal.dtx and rebuttal.ins.

\endpreamble

\generate{
  \file{rebuttal.sty}{\from{rebuttal.dtx}{package}}
  \file{rebuttal.tex}{\from{rebuttal.dtx}{rebuttal-comments}}
  \file{01-scrlttr2.tex}{\from{rebuttal.dtx}{example1}}
  \file{02-article.tex}{\from{rebuttal.dtx}{example2}}
}

\obeyspaces
\Msg{*************************************************************}
\Msg{*                                                           *}
\Msg{* To finish the installation you have to move the following *}
\Msg{* file into a directory searched by TeX:                    *}
\Msg{*                                                           *}
\Msg{*     rebuttal.sty                                          *}
\Msg{*                                                           *}
\Msg{* To produce the documentation run the file rebuttal.dtx    *}
\Msg{* through LaTeX.                                            *}
\Msg{*                                                           *}
\Msg{* Happy TeXing!                                             *}
\Msg{*                                                           *}
\Msg{*************************************************************}

\endbatchfile

\end{document}
%</example1>
%    \end{macrocode}
%
%    \begin{macrocode}
%<*example2>
\documentclass{article}
\usepackage{rebuttal}
\usepackage[colorlinks]{hyperref}
\usepackage{lipsum}
\usepackage{cleveref}

\begin{document}
\section{Test}
%%
%% Copyright (C) 2024 by Sergiu Deitsch
%%
%% This work may be distributed and/or modified under the
%% conditions of the LaTeX Project Public License, either version 1.3
%% of this license or (at your option) any later version.
%% The latest version of this license is in
%%   http://www.latex-project.org/lppl.txt
%% and version 1.3 or later is part of all distributions of LaTeX
%% version 2005/12/01 or later.
%%

\input docstrip.tex

\keepsilent
\askforoverwritefalse

\usedir{tex/latex/rebuttal}

\preamble

Copyright (C) 2018-2021, 2024 Sergiu Deitsch

This work may be distributed and/or modified under the
conditions of the LaTeX Project Public License, either version 1.3
of this license or (at your option) any later version.
The latest version of this license is in
  http://www.latex-project.org/lppl.txt
and version 1.3 or later is part of all distributions of LaTeX
version 2005/12/01 or later.

This work has the LPPL maintenance status `maintained'.

The Current Maintainer of this work is Sergiu Deitsch.

This work consists of the files rebuttal.dtx and rebuttal.ins.

\endpreamble

\generate{
  \file{rebuttal.sty}{\from{rebuttal.dtx}{package}}
  \file{rebuttal.tex}{\from{rebuttal.dtx}{rebuttal-comments}}
  \file{01-scrlttr2.tex}{\from{rebuttal.dtx}{example1}}
  \file{02-article.tex}{\from{rebuttal.dtx}{example2}}
}

\obeyspaces
\Msg{*************************************************************}
\Msg{*                                                           *}
\Msg{* To finish the installation you have to move the following *}
\Msg{* file into a directory searched by TeX:                    *}
\Msg{*                                                           *}
\Msg{*     rebuttal.sty                                          *}
\Msg{*                                                           *}
\Msg{* To produce the documentation run the file rebuttal.dtx    *}
\Msg{* through LaTeX.                                            *}
\Msg{*                                                           *}
\Msg{* Happy TeXing!                                             *}
\Msg{*                                                           *}
\Msg{*************************************************************}

\endbatchfile

\end{document}
%</example2>
%    \end{macrocode}
%
%    \begin{macrocode}
%<*rebuttal-comments>

The following pages contain a list of editor's and reviewers' comments followed
by our replies. The comments are sequentially numbered and associated either
with the editor or the corresponding reviewer. The replies may contain
references to changes in the original manuscript which are identified by a label
consisting of a running number and followed by the label of the original comment
in parentheses. The label back references the original reviewer comment within
the manuscript. For instance, the reference \textbf{C2~(1.3)}, which is typeset
in the manuscript margin, refers to the second change stemming from the third
comment of the first reviewer.

\begin{rebuttal}[Editor's Comments]
  \begin{comment}
  The topic of this manuscript is of interest. We concur with the summary review
  of the associate editor and the reviews of the referees. The authors should
  thoroughly address all of the comments, below, in a revised manuscript.
  \end{comment}

\end{rebuttal}

\begin{rebuttal}[Specific Comments]
  \begin{comment}
  \label{c:specific}
  You need to include figure captions that make your figures intelligible.
  See~\cref{a:foo} and \cref{c:foo2}.
  \end{comment}

  \begin{answer}
    The reviewer is right. We have reworked the figure captions to make the
    figures more independent of the text. Furthermore, check if lists are
    typeset correctly:
    \begin{itemize}
      \item One
      \item Two
        \begin{enumerate}
          \item Three
          \item Four
        \end{enumerate}
    \end{itemize}
  \end{answer}

  \begin{comment}
    \label{c:c2}
    Are you going to pay for color in the print article? If not, the lines are
    not intelligible in Figures 2, 5, 11. In addition to color coding for the
    digital version, you could make lines with distinct patterns (dots, dashes,
    etc.) so they are distinct in monochrome presentations.
  \end{comment}

  \begin{answer}
    We agree with this point and we are sorry that we missed this in the first
    version of the article. The article will appear as an online only version
    and is part of the special issue associated with this year’s CT Meeting
    conference. Thus there will be no additional cost associated with color
    figures. Nonetheless, we agree that all figures need to be correct in
    grey-scale as some readers might print the article. According to the
    reviewer’s comment, we have adjusted the figures by using different patterns
    and gray levels.

    (Refer also to \cref{c:specific} for more information.)
  \end{answer}
\end{rebuttal}

\lipsum[1] \addition[label=a:foo,ref=c:specific]{\lipsum[2]} \lipsum[3]

\lipsum[4] \deletion[label=a:foo1,ref=c:c2]{An unintended addition.}
and a \change[label=c:foo2,ref=c:c2]{from}{to} change. And another addition
\addition[label=a:foo3,ref=c:c2]{here}.
Also \addition[label=a:foo4,ref=c:c2]{here}.
Maybe \addition[label=a:foo5,ref=c:c2]{here}.
Additionally \addition[label=a:foo6,ref=c:c2]{here}.
Additionally \addition[label=a:foo7,ref=c:c2]{here}.

\begin{additionenv}[label=d:par,ref=c:c2]
  \begin{itemize}
    \item \lipsum[1]
  \end{itemize}
  \lipsum[2]
\end{additionenv}

Additional notes:
\begin{itemize}
  \item Multiple source references can be
    \addition[label=a:foo8,ref={c:specific,c:c2}]{specified}.
  \item If the references do not fit into a single line in the margin, you may
    force a \addition[label=a:break,ref={c:specific,c:c2},break]{break}.
\end{itemize}
%</rebuttal-comments>
%    \end{macrocode}
%
% \Finale

% \endinput

% vim: ft=tex
